\begin{table}[H]

\caption{Self-Reported Hispanic identity and Charles and Guryan (2012) Prejeduice Measure Among Second Generation Hispanic Immigrants: By Parental Type (State)\label{regtab-byparent-cg-02}}
\centering
\resizebox{\linewidth}{!}{
\begin{threeparttable}
\begin{tabular}[t]{lcccc}
\toprule
\multicolumn{1}{c}{Parents Type} & \multicolumn{1}{c}{All} & \multicolumn{1}{c}{HH} & \multicolumn{1}{c}{HW} & \multicolumn{1}{c}{WH} \\
\cmidrule(l{3pt}r{3pt}){1-1} \cmidrule(l{3pt}r{3pt}){2-2} \cmidrule(l{3pt}r{3pt}){3-3} \cmidrule(l{3pt}r{3pt}){4-4} \cmidrule(l{3pt}r{3pt}){5-5}
  & \specialcell{(1) \\ $H^2$} & \specialcell{(2) \\ $H^2$} & \specialcell{(3) \\ $H^2$} & \specialcell{(4) \\ $H^2$}\\
\midrule
Bias & \num{-0.09}* & \num{-0.05} & \num{-0.14} & \num{-0.21}**\\
 & (\num{0.05}) & (\num{0.04}) & (\num{0.09}) & (\num{0.10})\\
Female & \num{0.01} & \num{0.01}* & \num{0.00} & \num{0.01}\\
 & (\num{0.00}) & (\num{0.00}) & (\num{0.02}) & (\num{0.02})\\
College Graduate: Mother & \num{-0.14}*** & \num{-0.06}*** & \num{-0.22}*** & \num{-0.15}***\\
 & (\num{0.02}) & (\num{0.01}) & (\num{0.03}) & (\num{0.04})\\
College Graduate: Father & \num{-0.17}*** & \num{-0.05}*** & \num{-0.25}*** & \num{-0.25}***\\
 & (\num{0.02}) & (\num{0.01}) & (\num{0.04}) & (\num{0.03})\\
\midrule
N & \num{105064} & \num{71440} & \num{18961} & \num{14663}\\
R squared & \num{0.223} & \num{0.018} & \num{0.108} & \num{0.115}\\
Mean & \num{0.87} & \num{0.96} & \num{0.7} & \num{0.66}\\
\bottomrule
\multicolumn{5}{l}{\rule{0pt}{1em}* p $<$ 0.1, ** p $<$ 0.05, *** p $<$ 0.01}\\
\end{tabular}
\begin{tablenotes}
\small
\item[1] \footnotesize{Each column is an estimation of equation (\ref{eq:identity_reg_bias}) by 
                      parents' type with Charles and Guryan (2012) Prejeduice measure. 
                      Charles and Guryan (2012) use the General Security Survey of the most common racial questions between 1970-2000 for their measure of prejeduice.
                      To use Charles and Guryan (2012) prejeduice measure, I merge the GSS index in 1996 with CPS 1995-1999 years. 
                      In other words, I merge CPS data with the residual prejeduice measure from 20 years before the survey.
                      I include controls for sex, quartic age, parental education.
                      Standard errors are clustered on the state level}
\item[2] \footnotesize{The samples include second-generation Hispanic children ages 17 and below who live in intact families. 
                      Native born second-generation Hispanic 
                      immigrant children with at least one parent born in a Spanish speaking 
                      country.}
\item[3] \footnotesize{Data source is the 1994-2020 Current Population Survey.}
\end{tablenotes}
\end{threeparttable}}
\end{table}
