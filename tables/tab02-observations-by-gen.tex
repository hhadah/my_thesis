\begin{table}[H]

\caption{Hispanic Self-identification by Generation \label{tab:hispbygen}}
\centering
\resizebox{\linewidth}{!}{
\fontsize{12}{14}\selectfont
\resizebox{\linewidth}{!}{
\begin{threeparttable}
\begin{tabular}[t]{>{}lcccc}
\toprule
  & Self-identify as Hispanic & Self-identify as non-Hispanic & \% Self-identify as Hispanic & \% Self-identify as non-Hispanic\\
\midrule
\textbf{1st Gen.} & 114657 & 5121 & 0.96 & 0.04\\
\textbf{2nd Gen.} & 712916 & 48534 & 0.94 & 0.06\\
\hspace{1em}\textbf{Hispanic on:} &  &  &  & \\
\hspace{1em}\hspace{1em}\textbf{Both Sides} & 516551 & 19318 & 0.96 & 0.04\\
\hspace{1em}\hspace{1em}\textbf{One Side} & 196365 & 29216 & 0.87 & 0.13\\
\addlinespace
\textbf{3rd Gen.} & 209206 & 45493 & 0.82 & 0.18\\
\hspace{1em}\textbf{Hispanic on:} &  &  &  & \\
\hspace{1em}\hspace{1em}\textbf{Both Sides} & 55401 & 2245 & 0.96 & 0.04\\
\hspace{1em}\hspace{1em}\textbf{One Side} & 52879 & 17371 & 0.75 & 0.25\\
\bottomrule
\end{tabular}
\begin{tablenotes}
\item[1] The samples include children ages 17 and below who live in intact families. First-generation Hispanic immigrant children that were born in a Spanish speaking county. Native born second-generation Hispanic immigrant children with at least one parent born in a Spanish speaking country. Finally, native born third-generation Hispanic immigrant children with native born parents and at least one grandparent born in a Spanish speaking country.
\item[2] Data source is the 2004-2021 Current Population Survey.
\end{tablenotes}
\end{threeparttable}}}
\end{table}
