\begin{table}[!h]

\caption{Number of Children from the Different Types of Parents \label{tab:mat1}}
\centering
\resizebox{\linewidth}{!}{
\begin{threeparttable}
\begin{tabular}[t]{>{}lcccc}
\toprule
\multicolumn{1}{c}{ } & \multicolumn{4}{c}{Perent's Type} \\
\cmidrule(l{3pt}r{3pt}){2-5}
  & \specialcell{White Father \\ White Mother} & \specialcell{White Father \\ Hispanic Mother} & \specialcell{Hispanic Father \\ White Mother} & \specialcell{Hispanic Father \\ Hispanic Mother}\\
\midrule
\textbf{Observations} & \specialcell{3,828,024\\(0.95)} & \specialcell{24,306\\(0.01)} & \specialcell{35,841\\(0.01)} & \specialcell{135,300\\(0.03)}\\
\bottomrule
\end{tabular}
\begin{tablenotes}
\item[1] The data is restricted to people interviewed between 1994 and 2019, also White, married, and between the ages of 18 and 65. I identify the ethnicity of a person's parents through the parent's place of birth. A parent is Hispanic if both her parents were born in a Spanish-speaking country. A parent is White if born parents were born in the United States.
\end{tablenotes}
\end{threeparttable}}
\end{table}
